\chapter{Einleitung}
Diese Datei enth�lt die Anleitung zur Nutzung der Vorlage f�r verschiedene
Typen von Arbeiten. Sie ist vorrangig f�r Studenten (und auch wissenschaftliche
Mitarbeiter) gedacht, welche ihre Arbeiten bzw. Publikationen mit \LaTeX{}
erstellen wollen. Dabei wurden auch die Richtlinien des Corporate Design
der Technischen Universit�t Chemnitz ber�cksichtigt, soweit sie sich
ohne gr��ere Probleme in \LaTeX{} realisieren lassen.

Diese Vorlage ist f�r folgende Dokumente konzipiert, kann aber bei geringen
Modifikationen auch dar�ber hinaus eingesetzt werden:
\begin{itemize}
\item Hausarbeiten
\item Studienarbeiten
\item Diplomarbeiten
\item Praktikumsberichte
\item Proseminare
\item Oberseminare
\item Hauptseminare
\item sonstige Seminare
\item Belege
\item Studien
\end{itemize}


\begin{figure}
	\centering
		\includegraphics[width=0.50\textwidth]{TU_Chemnitz_positiv_gruen}
	\caption{Testbild}
	\label{fig:TU_Chemnitz_positiv_gruen}
\end{figure}

In den folgenden Kapiteln dieser Anleitung wird ein �berblick �ber die
Verwendung der Vorlage, Zweck der Dateien und typische Anwendungsf�lle
gegeben. Referenzierte Abbildung \ref{fig:TU_Chemnitz_positiv_gruen} auf Seite \pageref{fig:TU_Chemnitz_positiv_gruen}.
\textbf{Hinweis:} Diese Anleitung ist \textbf{keine} Einf�hrung
in \LaTeX. Dazu sei auf das Kursangebot des URZ bzw. auf weiterf�hrende
Literatur verwiesen. Auch erhebt diese Vorlage \textbf{nicht} den Anspruch,
da� jedes damit erstellte Dokument innerhalb der TU-Chemnitz grunds�tzlich
in Form, Umfang und Aufbau anerkannt wird. Studenten sollten dies 
grunds�tzlich vor der Verwendung anhand der f�r sie g�ltigen Studien- 
und Pr�fungsordnung pr�fen und dar�ber hinaus mit dem f�r sie 
zust�ndigen Professor bzw. Betreuer kl�ren.



